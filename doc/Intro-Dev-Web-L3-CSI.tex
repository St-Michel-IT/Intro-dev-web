\documentclass{beamer}
\RequirePackage{luatex85}
\include{../style/cours-style.sty}

% Title
\title{Introduction au Développement Web - Bachelor CSI}
\author{Christophe Brun}
\institute{Campus Saint-Michel IT}
\date{25 juin 2024}
\beamertemplatenavigationsymbolsempty

\titlegraphic{
    \bigbreak
    \includegraphics[width=2cm]{image/logo-papit}
    \includegraphics[width=2cm]{image/logo-campus-saint-michel-it}
}
\begin{document}

    \begin{frame}
        \titlepage
        \bigbreak
        \centering
        \url{https://github.com/St-Michel-IT/Intro-dev-web}
    \end{frame}

    \begin{frame}{Table des matières}
        \tableofcontents
    \end{frame}


    \section{Programme du module}\label{sec:programme-du-module}
    \begin{frame}{Introduction au Développement Web}{Compétences}
        Programme officiel du module~:
        \begin{itemize}
            \item Réaliser une interface HTML/CSS
            \item Lire/Écrire/Modifier des données dans une base de données avec PHP
            \item Gérer un formulaire
        \end{itemize}
        \bigbreak
        Des remarques~?
    \end{frame}


    \section{Évaluation}\label{sec:evaluation}
    \begin{frame}{Évaluation}
        \begin{itemize}
            \item 60 \% sur les projets développés au cours du module.
            \begin{itemize}
                \item Basée en partie sur les commits des développements pour comprendre facilement l'évolution du code.
                \item Les exercices doivent être terminés dans les temps et les livrables dans Teams.
            \end{itemize}
            \item 40 \% sur une évaluation écrite finale.
        \end{itemize}
    \end{frame}

    \begin{frame}{Intervenant sur le module Introduction au Développement Web}{Christophe Brun, conseil en développement informatique}

        \begin{columns}
            \column{0.7\textwidth}
            \begin{itemize}
                \item 2\textsuperscript{nde} année d'intervenant à Saint-Michel \emoji{star-struck}.

                \item 7 ans de conseil en développement au sein d'SSII~.

                \item 7 ans de conseil en développement à mon compte \href{https://papit.fr}{PapIT}.

                \item Passionné~!
                \bigbreak
                \begin{columns}
                    \column{0.5\textwidth}
                    \centering
                    \includegraphics[width=3cm]{image/logo-uppa}
                    \column{0.5\textwidth}
                    \centering
                    \includegraphics[width=3cm]{image/logo-universite-bordeaux}
                \end{columns}
            \end{itemize}
            \column{0.3\textwidth}
            \centering
            \includegraphics[width=5cm]{image/trombine-christophe}
        \end{columns}
    \end{frame}


    \section{Les ressources du Web}\label{sec:ressources}

    \begin{frame}{Les ressources du Web}{Dvisions en briques}
        \begin{columns}
            \column{0.6\textwidth}
            1 des 4 règles pour la direction de l'esprit de Descartes~: \textquote{Diviser chacune des difficultés que j'examinerais, en autant de parcelles qu'il se pourrait et qu'il serait requis pour les mieux résoudre.}.
            \column{0.4\textwidth}
            \centering
            \includegraphics[width=6cm]{image/Descartes}
        \end{columns}
    \end{frame}

    \begin{frame}{Les ressources du Web}{Analyse des composants, exercice \execcounterdispinc{}}
        Listez toutes les ressources du Web que vous connaissez.
        \begin{itemize}
            \item ...
        \end{itemize}
        \bigbreak
        \begin{columns}
            \column{0.5\textwidth}
            Restituez-les dans un schéma au formalisme libre (Draw.io est conseillé\ldots).

            \textbf{Prenez le temps de bien nommer chaque concept.}
            \bigbreak
            À présenter au tableau~!
            \column{0.5\textwidth}
            \centering
            \includegraphics[width=5cm]{image/computer-n-web-ressources}
        \end{columns}
    \end{frame}


    \section{Les protocoles du Web}\label{sec:protocoles}

    \begin{frame}{Les protocoles}{Introductions\footnote{\label{sendbird-protocole}Protocoles de communication WebSocket vs. HTTP, \url{https://sendbird.com/fr/developer/tutorials/websocket-vs-http-communication-protocols}}}
        Si on définit le Web comme les applications entre un navigateur et un serveur Web.
        \begin{dangercolorbox}
            Ce qui est très réducteur, car il ne prend pas en compte les applications mobiles, les API, les applications desktop, mail, etc.

            Mais c'est l'unique sujet de ce module\ldots
        \end{dangercolorbox}
        Quels sont les protocoles qui permettent à ces applications de communiquer~?
        \pause
        \bigbreak
        Il n'y en a que 2~!
        \begin{itemize}
            \item HTTP(S)
            \item Websocket
        \end{itemize}
    \end{frame}

    \begin{frame}{Les protocoles}{Introduction\cref{sendbird-protocole}}
        HTTP, par contre, est un protocole de communication semi-duplex, qui existe depuis un certain temps et constitue la base du Web depuis ses débuts.

        HTTP date de 1989, inventé au CERN par Tim Berners-Lee\footnote{The birth of the Web, \url{https://home.cern/science/computing/birth-web}}.
        \bigbreak
        WebSocket, un protocole de communication full-duplex, est relativement plus récent et convient mieux aux applications en temps réel (ou presque\ldots) comme le live chat dans les applications mobiles, les notifications et ou les appels audio ou vidéo.

        Websocket a été standardisé en 2011\footnote{The WebSocket Protocol, \url{https://datatracker.ietf.org/doc/html/rfc6455}}.
        Il est géré par le JavaScript du navigateur.
    \end{frame}

    \subsection{WebSocket}\label{subsec:websocket}

    \begin{frame}{Les protocoles}{Websocket, exercice \execcounterdispinc{}}
        \begin{itemize}
            \item Que veut dire full-duplex~?
            \item Y-a-t-il une contradiction avec une architecture client/serveur~?
            \item Citer un autre protocol full-duplex.
            \item Citer un protocol qui \textbf{n'est pas} full-duplex.
        \end{itemize}
        \bigbreak
        \centering
        \includegraphics[width=5cm]{image/homework}
    \end{frame}

    \begin{frame}{Les protocoles}{Websocket\cref{sendbird-protocole}}
        \centering
        \includegraphics[width=12cm]{image/tutorial-websocket-protocol-chart}
    \end{frame}

    \begin{frame}{Les protocoles}{Websocket côté client (Navigateur)\footnote{\label{mozilla-websocket}WebSockets, \url{https://developer.mozilla.org/fr/docs/Web/API/WebSockets_API}}}
        \begin{columns}
            \column{0.5\textwidth}
            \centering
            \includegraphics[width=3.3cm]{image/client-support} \\ Client \\
            \column{0.5\textwidth}
            \centering
            \includegraphics[width=6.5cm]{image/kids-on-the-phone}
        \end{columns}
    \end{frame}

    \begin{frame}{Les protocoles}{Websocket côté serveur (liste non exhaustive !)\cref{mozilla-websocket}}
        \begin{scriptsize}
            \begin{itemize}

                \item \href{https://github.com/uWebSockets/uWebSockets}{µWebSockets}~: Déclinaison plus légère et plus performante de WebSocket et écrite en \href{https://isocpp.org/}{C++11} et en \href{https://nodejs.org/fr/}{Node.js}.

                \item \href{https://github.com/ClusterWS/ClusterWS}{ClusteWS~}: Framework léger, rapide et puissant qui permet de construire des applications en \href{https://nodejs.org/fr/}{Node.js}.

                \item \href{http://socket.io}{Socket.IO}~: API WebSocket puissante et multiplateforme en \href{https://nodejs.org}{Node.js}.

                \item \href{https://socketcluster.io/\#!/}{SocketCluster}~: Framework open source en temps réel en \href{https://nodejs.org}{Node.js}.

                \item \href{https://nodejs.org}{Node.js}.

                \item \href{https://www.totaljs.com/}{Total.js}~: FrameWork pour web application en \href{https://nodejs.org}{Node.js}.

                \item \href{https://www.npmjs.com/package/faye-websocket}{Faye}~: Combine WebSocket(bidirectionnelle) et EventSource(unidirectionnelle), côté serveur et côté client en \href{https://nodejs.org}{Node.js}.

                \item \href{https://signalr.net/}{SignalR}~: SignalR est une nouvelle bibliothèque pour les développeurs \href{https://dotnet.microsoft.com/apps/aspnet}{ASP.NET}.

                \item \href{https://caddyserver.com/docs/websocket}{Caddy}~: Serveur web capable de créer des WebSockets serveur/proxy(stdin/stdout, echo, cat, \ldots).

                \item \href{https://github.com/websockets/ws}{ws}~: La plus populaire des WebSockets pour client \& serveur en \href{https://nodejs.org}{Node.js}.

                \item \href{https://github.com/bigstepinc/jsonrpc-bidirectional}{jsonrpc-bidirectional}~: Implémentation de JSON-RPC 2.0 sur WebSocket.

                \item \href{https://github.com/ninenines/cowboy}{cowboy}~: Cowboy est un petit serveur HTTP rapide et moderne pour Erlang/OTP basé sur WebSocket.

                \item \href{https://zeromq.org}{ZeroMQ}~: ZeroMQ est une bibliothèque de fonctions pour transporter des messages avec divers protocoles IPC, TCP, UDP, TIPC, diffusion de groupe et WebSocket.

            \end{itemize}
        \end{scriptsize}
    \end{frame}

    \begin{frame}{Les protocoles}{Websocket, exercice \execcounterdispinc{}}
        Dans un groupe de deux ou trois~:
        \begin{itemize}
            \item Développer une page HTML avec un JavaScript se connectant à un serveur WebSocket avec la techno/librairie de votre choix.
            \item Développer un serveur WebSocket avec la techno/librairie de votre choix.
            \item Les livrables, codes source \textbf{commentés} et captures d'écran, sont à déposer dans Teams.
        \end{itemize}
        \begin{dangercolorbox}
            À la vue de l'omniprésence de ce protocole sur de nombreuses technos et de la bonne compatibilité des méthodes.
            Est-il nécessaire d'utiliser des librairies (dépendances logicielles)~?
        \end{dangercolorbox}
    \end{frame}

    \subsection{HTTP}\label{subsec:http}

    \begin{frame}{Les protocoles}{HTTP\cref{sendbird-protocole}}
        \centering
        \includegraphics[width=12cm]{image/Tutorial-HTTP-connection-chart}
    \end{frame}

    \subsection{WebSocket VS HTTP}\label{subsec:ws-vs-http}

    \begin{frame}{Les protocoles}{HTTP\cref{sendbird-protocole}}
        \centering
        \includegraphics[width=12cm]{image/Tutorial-WebSocket-vs.-HTTP-communication-diagram}
    \end{frame}

    \begin{frame}{HTTP}{Introduction aux méthodes HTTP\footnote{\label{mozilla-http-methods}Méthodes de requête HTTP, \url{https://developer.mozilla.org/fr/docs/Web/HTTP/Methods}}}
        \begin{itemize}
            \item HTTP définit des méthodes de requête pour interagir avec les ressources.
            \item Ces méthodes sont souvent appelées \textit{verbes HTTP}.
            \item Les méthodes partagent parfois des fonctionnalités comme la \textit{sécurité}, l'\textit{idempotence}, ou la possibilité d'être \textit{mise en cache}.
        \end{itemize}
    \end{frame}

    \begin{frame}{HTTP}{Méthodes courantes\cref{mozilla-http-methods}}
        \begin{itemize}
            \item \lstinline{GET} : Récupère une ressource, sans modifier l'état du serveur (idempotence).
            Autorisée dans les \lstinline{<form>} HTML.
            \item \lstinline{HEAD} : Similaire à \lstinline{GET}~, sans le corps de la réponse.
            \item \lstinline{POST} : Envoie des données, créant ou modifiant une ressource.
            Autorisée dans les \lstinline{<form>} HTML.
            \item \lstinline{PUT} : Remplace la ressource avec les données envoyées.
            \item \lstinline{DELETE} : Supprime une ressource.
        \end{itemize}
        \begin{dangercolorbox}
            \lstinline{POST} et \lstinline{GET} sont compatibles avec les formulaires, mais à quoi donc servent alors les autres~?
        \end{dangercolorbox}
    \end{frame}

    \begin{frame}{HTTP}{Autres méthodes\cref{mozilla-http-methods}}
        \begin{itemize}
            \item \lstinline{CONNECT} : Établit un tunnel vers le serveur cible.
            \item \lstinline{OPTIONS} : Renvoie les options de communication avec la ressource.
            \item \lstinline{TRACE} : Effectue un test aller-retour sur le chemin suivi.
            \item \lstinline{PATCH} : Applique des modifications partielles à une ressource.
        \end{itemize}
    \end{frame}

    \begin{frame}{HTTP}{Les URLs\footnote{Top REST API URL naming convention standards, \url{https://www.theserverside.com/video/Top-REST-API-URL-naming-convention-standards}}}
        \begin{tiny}
            Guidelines API REST~:
            \begin{itemize}
                \item Utilisez uniquement des lettres minuscules dans les URLs des API RESTful.
                \item Pour les espaces, utilisez kebab-case, et non snake\_case ou des espaces.
                \item Construisez les URIs avec des noms, pas des verbes.
                \item Utilisez les méthodes HTTP appropriées pour effectuer une opération.
                \item Les appels d'API REST qui retournent une collection doivent être au pluriel.
                \item Une URL qui retourne un résultat unique doit être au singulier.
                \item N'incluez pas d'extensions de fichier.
                \item Utilisez les en-têtes pour garder les URIs propres.
                \item N'identifiez pas les opérations de création, lecture, mise à jour et suppression (CRUD) dans l'URL~.
                \item Structurez les URIs librement selon la hiérarchie de votre modèle de données.
                \item Utilisez des paramètres de requête pour le filtrage et la recherche.
                \item Ne dévoilez pas le fonctionnement interne de votre architecture.
                \item Faites des URLs courtes, intuitives et lisibles.
                \item Protégez-vous contre les injections SQL~.
                \item Incluez la version de l'API REST à la base de l'URI~.
            \end{itemize}
        \end{tiny}
    \end{frame}

    \begin{frame}{HTTP}{Introduction à Content-Type\footnote{\label{mozilla-content-type}Content-Type, \url{https://developer.mozilla.org/fr/docs/Web/HTTP/Headers/Content-Type}}}
        \begin{itemize}
            \item \textbf{Content-Type} : Indique le type MIME de la ressource.
            \item Utilisé dans les réponses pour informer le client du type de contenu renvoyé.
            \item Utilisé dans les requêtes POST/PUT pour indiquer au serveur le type de données envoyées.
        \end{itemize}
        \bigbreak
        Pourquoi pas dans les GET~?

        La donnée est toujours dans l'URL et non dans le corps de la requête comme pour le POST.
        C'est donc inutile.
    \end{frame}

    \begin{frame}{HTTP}{Directives de Content-Type\cref{mozilla-content-type}}
        \begin{itemize}
            \item \lstinline{media-type} : Type MIME des données.
            \item \lstinline{charset} : Encodage des caractères.
            \item \lstinline{boundary} : Limites pour les entités multipart.
        \end{itemize}
    \end{frame}

    \begin{frame}[fragile]{HTTP}{Exemple : Formulaires HTML\cref{mozilla-content-type}}
        \begin{lstlisting}[language=HTML]
<form action="/" method="post" enctype="multipart/form-data">
  <input type="text" name="description" value="du texte" />
  <input type="file" name="monFichier" />
  <button type="submit">Envoyer</button>
</form>
        \end{lstlisting}
        \begin{itemize}
            \item Requête POST envoyée avec Content-Type multipart/form-data.
        \end{itemize}
        \begin{lstlisting}[language=HTML]
<form action="/recherche" method="get">
    <input type="text" name="q" value="recherche" />
    <button type="submit">Rechercher</button>
</form>
        \end{lstlisting}
        \begin{itemize}
            \item Requête GET envoyée avec Content-Type application/x-www-form-urlencoded.
        \end{itemize}

    \end{frame}


    \begin{frame}{Tableau des codes de statut HTTP\footnote{Codes de réponse HTTP, \url{https://developer.mozilla.org/fr/docs/Web/HTTP/Status}}}
        \begin{tabular}{|p{1.5cm}|p{9.5cm}|}
            \hline
            \textbf{Code} & \textbf{Description}                                                         \\
            \hline
            100-199       & Réponses informatives (e.g. 100 Continue, 101 Switching Protocols)           \\
            \hline
            200-299       & Succès (e.g. 200 OK, 201 Created)                                            \\
            \hline
            300-399       & Redirections (e.g. 301 Moved Permanently, 302 Found)                         \\
            \hline
            400-499       & Erreurs client (e.g. 400 Bad Request, 404 Not Found, 405 Method Not Allowed) \\
            \hline
            500-599       & Erreurs serveur (e.g. 500 Internal Server Error, 503 Service Unavailable)    \\
            \hline
        \end{tabular}
    \end{frame}


    \begin{frame}{HTTP}{L'approche frameworkless}
        La plupart des langages comme Python ou Java viennent avec des libraires pour traiter les URL, les méthodes HTTP, les requêtes, \textit{etc}.
        \bigbreak
        \begin{itemize}
            \item Pourquoi, comment, expliquer le succès des frameworks webs~?
            \item  Quel(s) désavantage(s) ont les librairies natives~?
            \item  Quel(s) désavantage(s) ont les frameworks~?
        \end{itemize}
        \bigbreak
        Exercice \execcounterdispinc{}~: Développer un serveur HTTP en Python sans librairie tierce à partir de ce tutoriel \url{https://python.doctor/page-python-serveur-web-creer-rapidement}~.

        Le compléter avec tous les éléments vus précédemment, content-type, méthodes HTTP, URL.
    \end{frame}

    \begin{frame}{HTTP}{Le serveur web de 25 lignes}
        \centering
        \includegraphics[width=12cm]{image/25-lines-server} \\ \url{https://www.youtube.com/watch?v=7GBlCinu9yg} \\
    \end{frame}


    \section{Frameworks}\label{sec:framewoks}


    \section{Définition d'un frameworks}\label{subsec:framewok-definition}

    \begin{frame}{Les Frameworks}{Qu'est-ce qu'un framework en programmation et en ingénierie~?\footnote{Qu'est-ce qu'un framework en programmation et en ingénierie ?, \url{https://aws.amazon.com/fr/what-is/framework/}}}
        \begin{small}
            En génie logiciel et en programmation, un framework est un ensemble de composants logiciels réutilisables qui permettent de développer de nouvelles applications plus efficacement.
            La réutilisation du développement et de la recherche existants est un principe essentiel dans tous les domaines de l'ingénierie.
            Par exemple, les ingénieurs électriciens utilisent des composants électroniques existants pour fabriquer de nouveaux appareils.
            Les fabricants de composants suivent des normes et des spécifications prédéterminées pour garantir l'utilisabilité des composants.
            \bigbreak
            De même, les frameworks logiciels contiennent des modules de code réutilisables basés sur des normes et protocoles logiciels spécifiques.
            Les frameworks peuvent également définir et appliquer certaines règles d'architecture logicielle ou certains processus métier, afin que de nouvelles applications puissent être développées de manière standardisée.
            \bigbreak
            Les allergiques aux anglicismes parlent de cadriciel.
        \end{small}
    \end{frame}

    \begin{frame}{Les Frameworks}{Qu'est-ce qu'un framework web?}
        On dénombre principalement 3 types de frameworks web~:
        \begin{itemize}
            \item \textbf{Back-End}, comme Django, Laravel, NextJS
            \item \textbf{Front-End orienté composant}, comme React, Angular Vue
            \item \textbf{Front-End CSS}, comme Bootstrap
        \end{itemize}
    \end{frame}

    \subsection{Framework front-end CSS}\label{subsec:framewok-css}

    \begin{frame}{Les Frameworks}{Framework front-end CSS}
        Utilisent le CSS pour modifier l'affichage par défaut des éléments HTML.

        Mais en plus, les classes qui peuvent être ajoutées aux balises HTML pour plus de flexibilité.

        Ils ne sont souvent pas uniquement CSS, mais contiennent également des éléments JavaScript pour des fonctionnalités plus dynamiques comme les pop-up.
        \bigbreak
        En plus de ce framework, du CSS propre à l'identité visuelle souhaitée peut venir surcharger celui du framework (au niveau des couleurs, typo, \textit{etc}).
        \bigbreak
        Ils sont souvent complémentaires d'un back-end ayant un moteur de templating HTML, car ils peuvent s'intégrer à ce dernier.
        \bigbreak
        Parmi des simples et populaires actuellement~:
        \begin{itemize}
            \item \textbf{Bootstrap 5}
            \item \textbf{Tail wind}
        \end{itemize}
    \end{frame}

    \begin{frame}[fragile]{Les Frameworks}{Framework front-end CSS, exemple de code}
        \begin{lstlisting}[language=HTML,basicstyle=\tiny\ttfamily]
<!DOCTYPE html>
<html lang="fr">
<head>
    <meta charset="UTF-8">
    <meta name="viewport" content="width=device-width, initial-scale=1.0">
    <title>Fournisseur - Fondation Alia</title>
    <link href="https://cdn.jsdelivr.net/npm/bootstrap@5.3.0-alpha1/dist/css/bootstrap.min.css" rel="stylesheet">
    <style>
        body {background-color: #f5f5f5;}
...
<body>
    <header>
        <nav class="navbar navbar-expand-lg navbar-light bg-light">
            <div class="container">
                <a class="navbar-brand" href="#">
                    <img src="https://fondationalia.fr/wp-content/themes/interlude-VSHA/images/Logo-header.svg"
                        alt="Logo de la fondation Alia" width="80"></a>
                <h1 class="text-center">Articles</h1>
            </div>
        </nav>
        <div class="container">
 ...
        <footer>&copy; 2024 Fondation Alia - Tous droits réservés</footer>
        <script src="https://cdn.jsdelivr.net/npm/bootstrap@5.3.0-alpha1/dist/js/bootstrap.bundle.min.js"></script>
</body>
</html>
        \end{lstlisting}
    \end{frame}

    \begin{frame}{Les Frameworks}{Framework front-end CSS, les tendances}
        \centering
        \includegraphics[width=11cm]{image/css-trends}
    \end{frame}

    \begin{frame}{Les Frameworks}{Frameworks front-end CSS, exercice \execcounterdispinc{}}
        À l'aide de Bootstrap 5 ou Tailwind, développer 2 pages HTML ayant un look and feel moderne et agréable à regarder~\emoji{face-with-spiral-eyes}~:
        \begin{itemize}
            \item Un formulaire qui liste des articles avec un bouton pour en ajouter un avec la balise HTML \lstinline{form}~.
            \item Une page de liste des articles avec la balise HTML \lstinline{table}~.
            \item Les 2 pages ont des header et footer identiques.
        \end{itemize}
        \bigbreak
        \begin{columns}
            \column{0.5\textwidth}
            \centering
            \includegraphics[width=6cm]{image/form-item}
            \column{0.5\textwidth}
            \centering
            \includegraphics[width=6cm]{image/list-items}
        \end{columns}
    \end{frame}

    \subsection{Framework front-end orienté composant}\label{subsec:framewok-component}

    \begin{frame}{Les Frameworks}{Framework front-end orienté composant}
        Les frameworks front-end orientés composants sont des librairies qui permettent de construire des interfaces utilisateur à partir de composants réutilisables.
        \bigbreak
        Les composants sont des éléments d'interface utilisateur autonomes, réutilisables et indépendants qui peuvent être assemblés pour créer des interfaces utilisateur plus complexes.
        \bigbreak
        Ils sont idéals en front-end d'une API.
        Et peuvent donc être complémentaire d'un back-end qui sert le JSON.
        Backend agnostique à un UI pouvant être utilisé par d'autres front-ends comme celui d'une application mobile.
    \end{frame}

    \begin{frame}{Les Frameworks}{Framework front-end orienté composant}
        Pour ne pas à avoir beaucoup de CSS, ils peuvent appeler une librairie CSS type Bootstrap.
        \bigbreak
        Un composant est un élément qui vient avec son HTML/CSS/JS ou TS propre.
        \bigbreak
        Parmi les plus populaires~:
        \begin{itemize}
            \item \textbf{React}
            \item \textbf{Angular}
            \item \textbf{Vue}
        \end{itemize}
    \end{frame}

    \begin{frame}[fragile]{Les Frameworks}{Framework front-end orienté composant}

        \begin{columns}
            \column{0.2\textwidth}
            \centering
            \includegraphics[width=2.5cm]{image/react-beginner-project-structure}
            \column{0.8\textwidth}
            \begin{dangercolorbox}
                La courbe d'apprentissage de ces derniers et plus raide que pour un framework CSS.
                Ils ne font pas partie du scope de cette matière.
            \end{dangercolorbox}
            Exemple d'un composant de type bouton~:
            \begin{lstlisting}[language=Java,basicstyle=\tiny\ttfamily]
import * as React from 'react';
const App = () => {
  const [isOpen, setOpen] = React.useState(false);
  const handleClick = () => {
    setOpen(!isOpen);
  };
  return (
    <div>
      <button type="button" onClick={handleClick}>
        Click Me
      </button>
      {isOpen && <div>Content</div>}
    </div>
  );
};
export default App;
            \end{lstlisting}
        \end{columns}
    \end{frame}

    \begin{frame}{Les Frameworks}{Framework front-end orienté composant, les tendances}
        \centering
        \includegraphics[width=11cm]{image/front-compo-trends}
    \end{frame}

    \subsection{Framework back-end}\label{subsec:framewok-back-end}

    \begin{frame}{Les Frameworks}{Framework back-end}
        Les frameworks back-end sont des librairies qui permettent de construire des applications web en fournissant des fonctionnalités de base autour du HTTP et du templating HTML.
        \bigbreak
        Ils sont souvent utilisés pour construire des API REST (JSON).
        Le moteur de templating permet un meilleur SEO (référencement naturel) qu'une UI générée \textit{server-side} car le rendu est plus rapide.
        \bigbreak
        Parmi les plus populaires~:
        \begin{itemize}
            \item \textbf{Django} en Python
            \item \textbf{Flask} en Python
            \item \textbf{FastAPI} en Python
            \item \textbf{Laravel} en PHP
            \item \textbf{Spring} en Java
            \item \textbf{NextJS} en JavaScript
        \end{itemize}
    \end{frame}

    \begin{frame}{Les Frameworks}{Framework back-end, Flask}
        \begin{small}
            Utilise Jinja2 comme moteur de templating, il peut donc servir du JSON ou du HTML à partir de ces templates.
            \bigbreak
            Il communique avec le serveur web grâce au WSGI\footnote{WSGI, \url{https://perso.liris.cnrs.fr/pierre-antoine.champin/2017/progweb-python/cours/cm1.html\#id2}}.
            \bigbreak
            Il vient avec un serveur de développement, mais en production, il est utilisé avec\footnote{Self-Hosted Options, \url{https://flask.palletsprojects.com/en/3.0.x/deploying/\#self-hosted-options}}~:
            \begin{itemize}
                \item \href{https://flask.palletsprojects.com/en/3.0.x/deploying/gunicorn/}{Gunicorn}
                \item \href{https://flask.palletsprojects.com/en/3.0.x/deploying/waitress/}{Waitress}
                \item \href{https://flask.palletsprojects.com/en/3.0.x/deploying/mod_wsgi/}{mod\_wsgi} de Apache
                \item \href{https://flask.palletsprojects.com/en/3.0.x/deploying/uwsgi/}{uWSGI} de Nginx
                \item \href{https://flask.palletsprojects.com/en/3.0.x/deploying/gevent/}{gevent}
                \item \href{https://flask.palletsprojects.com/en/3.0.x/deploying/eventlet/}{eventlet}
                \item \href{https://flask.palletsprojects.com/en/3.0.x/deploying/asgi/}{ASGI}
            \end{itemize}
        \end{small}
    \end{frame}

    \begin{frame}{Les Frameworks}{Framework back-end, Flask\footnote{Project Layout, \url{https://flask.palletsprojects.com/en/2.3.x/tutorial/layout/}}}
        C'est un des frameworks les plus simples à prendre en main.
        Voir le \href{https://www.digitalocean.com/community/tutorials/how-to-make-a-web-application-using-flask-in-python-3}{tutoriel Flask de Digital Ocean}.
        \bigbreak
        \begin{columns}
            \column{0.3\textwidth}
            \centering
            \includegraphics[width=3cm]{image/flask-project-structure}
            \column{0.7\textwidth}
            On y trouve~:
            \begin{itemize}
                \item Les ressources statiques HTML, CSS, JS dans le dossier \lstinline{static/}~.
                \item Les tests dans \lstinline{tests/}~.
                \item Les templates HTML dans \lstinline{templates/}~.
                \item Les applications Flask \lstinline{auth.py} et \lstinline{blog.py} avec les routes.
                \item \ldots
            \end{itemize}
        \end{columns}
    \end{frame}
    
    \begin{frame}{Les Frameworks}{Framework back-end, les tendances}
        \centering
        \includegraphics[width=11cm]{image/back-end-trends}
    \end{frame}
    
    \begin{frame}{Les Frameworks}{Framework back-end, exercices}
        Exercice \execcounterdispinc{}, avec Flask, \textit{Donner vie} aux fichiers HTML développés précédemment~:
        \bigbreak
        \begin{columns}
            \column{0.6\textwidth}
            \begin{itemize}
                \item Utiliser les templates pour n'écrire qu'une fois le header et le footer et les fichiers statiques pour le CSS et JS.
                Configurer Flask pour utiliser ces dossiers.
                \item Ne pas développer de base de données complexe.
                Un dictionnaire pour stocker les données du formulaire suffira.
            \end{itemize}
            \column{0.4\textwidth}
            \centering
            \includegraphics[width=5cm]{image/html-alive}
        \end{columns}
    \end{frame}

    \begin{frame}{Les Frameworks}{Framework back-end, exercices}
        Exercice \execcounterdispinc{}, avec Flask, intégrer les méthodes d'appel à la BDD du script \url{https://github.com/St-Michel-IT/Intro-dev-web/blob/master/customer_database.py}~:
        \begin{itemize}
            \item Sous forme d'API JSON, inspirée de la nomenclature et des bonnes pratiques REST.
            \item Utiliser le verbe HTTP adapté à la fonction de la méthode appelée, (car c'est une API).
            \item Il n'y a donc que les routes à développer, elle retourne un JSON avec la méthode Flask \lstinline{flask.jsonify}
            \item API donc pas de dossier \lstinline{static}~, ni même de template dans ce cas simple.
            \item Certaines méthodes ne retournant rien, utiliser les exceptions pour retourner le bon status HTTP.
        \end{itemize}
    \end{frame}


    \section{Licence CC}\label{sec:licence}

    \begin{frame}{Licence}{Licence Creative Commons}
        Support de cours sous licence Creative Commons BY-NC-ND~.
        \bigbreak
        Vous pouvez donc partager, copier, distribuer le document.
        \bigbreak
        Attribution requise à PapIT SASU - Pas d’utilisation commerciale - Pas de modification
        \bigbreak
        \centering
        \includegraphics[width=5cm]{image/by-nc-nd-logo}
    \end{frame}
\end{document}
